\documentclass[5pt,a4paper]{article}

\usepackage{amsmath} % Essential math and alignment (use `&` to align operators')
% Google for more info
\usepackage{amssymb} % Essential symbols for sets and stuff

\usepackage[margin=0.5in]{geometry} % Essential document config

\usepackage{multicol} % Imports the column
\setlength{\columnseprule}{0.4pt} % Divide size

\begin{document}
\begin{multicols}{3}

    \section{RAM Model}
    \subsection{Memory}
    Infinite sequence of cells, contains $w$ bits. Every cell has an address starting at 1
    \subsection{CPU}
    32 registers of width $w$ bits.
    \subsubsection{Operations}
    Set value to register (constant or from other register). Take two integers from other registers and store the result of; $a+b$, $a-b$, $a\cdot b$, $a/b$. Take two registers and compare them; $a<b$, $a=b$, $a>b$. Read and write from memory.
    \subsection{Definitions}
    An algorithm is a set of atomic operations. It's cost is is the number of atomic operations. A word is a sequence of $w$ bits
    \section{Worst-case}
    Worst-case cost of an algorithm is the longest possible running time of input size $n$
    \section{Dictionary search}
    let $n$ be register 1, and $v$ be register 2\\
    register $left\rightarrow1$, $right\rightarrow1$\\
    while $left\leq right$\\
    \indent register $mid\rightarrow(left+right)/2$\\
    \indent if the memory cell at address $mid=v$ then\\
    \indent\indent return yes\\
    \indent else if memory cell at address $mid>v$ then\\
    \indent\indent $right=mid-1$\\
    \indent else\\
    \indent\indent $left=mid+1$\\
    return no\\\\
    Worst-case time: $f_2(n)=2+6\log_2n$


\end{multicols}

\end{document}

