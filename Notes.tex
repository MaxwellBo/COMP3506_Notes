\documentclass[5pt,a4paper]{article}

\usepackage{amsmath} % Essential math and alignment (use `&` to align operators')
% Google for more info
\usepackage{amssymb} % Essential symbols for sets and stuff

\usepackage[margin=0.5in]{geometry} % Essential document config

\usepackage{multicol} % Imports the column
\setlength{\columnseprule}{0.4pt} % Divide size

\begin{document}
\begin{multicols}{3}

\section{Week 3}

\begin{itemize}
\item If $f(n) = O(g(n))$, there exist constants $c_1$ and $c_2$ such that $f(n) \leqslant c_2 \cdot g(n)$ holds for all $n \geqslant c_2$.
\item If $f(n) = O(g(n))$, we have $\lim_{n \to \infty} \dfrac{f_1(n)}{g_1(n)} = c$ = some constant $c$. 
\end{itemize}

\section{Week 3 - Extra}

\begin{itemize}
\item When using `Direction 1: Constant Finding' setting $c_1$, always set it to match the coefficent on the LHS so that you can cancel.
\item When trying to get a contradiction, try and isolate an $x \cdot c_1$ on the RHS, where $x \in \mathbb{Z}$, such that an expression that contains $n$ is $\leqslant xc_1$
\item Make judicious use of the $max$ function when adding functions together
\item If $f_1(n) + f_2(n) \leqslant c_1 \cdot g_1(n) +c'_1 \cdot g_2(n) \leqslant max\{c_1 , c'_1 \} \cdot (g_1(n) + g_2(n))$, for all $n \geqslant max\{c_2, c'_2\}$.

\end{itemize}

\end{multicols}

\end{document}

