\documentclass[10pt,a4paper]{article}

\usepackage{amsmath} % Essential math and alignment (use `&` to align operators')
% Google for more info
\usepackage{amssymb} % Essential symbols for sets and stuff

\usepackage[margin=0.3in]{geometry} % Essential document config

\usepackage[compact]{titlesec}
%\usepackage{sectsty}
%\sectionfont{\fontsize{7}{9}\selectfont}
%\subsectionfont{\fontsize{6}{8}\selectfont}
%\subsubsectionfont{\fontsize{5}{7}\selectfont}

\usepackage{multicol} % Imports the column
\setlength{\columnseprule}{0.4pt} % Divide size

\begin{document}
\begin{multicols}{3}

    \small

    \section{RAM Model}
    \subsection{Memory}
    Infinite sequence of cells, contains $w$ bits. Every cell has an address starting at 1
    \subsection{CPU}
    32 registers of width $w$ bits.
    \subsubsection{Operations}
    Set value to register (constant or from other register). Take two integers from other registers and store the result of; $a+b$, $a-b$, $a\cdot b$, $a/b$. Take two registers and compare them; $a<b$, $a=b$, $a>b$. Read and write from memory.
    \subsection{Definitions}
    An algorithm is a set of atomic operations. It's cost is is the number of atomic operations. A word is a sequence of $w$ bits
    \section{Worst-case}
    Worst-case cost of an algorithm is the longest possible running time of input size $n$
    \section{Dictionary search}
    let $n$ be register 1, and $v$ be register 2\\
    register $left\rightarrow1$, $right\rightarrow1$\\
    while $left\leq right$\\
    \indent register $mid\rightarrow(left+right)/2$\\
    \indent if the memory cell at address $mid=v$ then\\
    \indent\indent return yes\\
    \indent else if memory cell at address $mid>v$ then\\
    \indent\indent $right=mid-1$\\
    \indent else\\
    \indent\indent $left=mid+1$\\
    return no\\\\
    Worst-case time: $f_2(n)=2+6\log_2n$

    \section{Big-O}
    We say that $f(n)$ grows asymptotically no faster than $g(n)$ if there is a constant $c_1>0$ such that $f(n)\leq c_1\cdot g(n)$ and holds for all $n$ at least a constant $c_2$. This is denoted by $f(n)=O(g(n))$.\\
    $\lim_{n\rightarrow\infty}\frac{f(n)}{g(n)}=c$ for some constant $c$
    \subsection{Example}
    $1000\log_2n=O(n)$,\\$n\neq O(10000\log_2n)$\\
    $\log_{b_1}n=O(\log_{b_2}n)$ for any constants $b_1>1$ and $b_2>1$.
    Therefore $f(n)=2+6\log_2n$ can be represented; $f(n)=O(\log n)$
    \section{Big-$\Omega$}
    If $g(n)=O(f(n))$, then $f(n)=\Omega(g(n))$ to indicate that $f(n)$ grows asymptotically no slower than $g(n)$. We say that $f(n)$ grows asymptotically no slower than $g(n)$ if $c_1>0$ such $f(n)\geq c_1\cdot g(n)$ for $n>c_2$; denoted by $f(n)=\Omega(g(n))$
    \section{Big-$\Theta$}
    If $f(n)=O(g(n))$ and $f(n)=\Omega(g(n))$, then $f(n)=\Theta(g(n))$ to indicate that $f(n)$ grows asymptotically as fast as $g(n)$

    \section{Sort}
    \subsection{Merge Sort}
    Divide the array into two parts, sort the individual arrays then combine the arrays together.
    $f(n)=O(n\log n)$.\\
    This is the fastest sorting time possible (apart from $O(n\log\log n)$
    \subsection{Counting Sort}
    A set S of n integers and every integer is in the range [1, U]. (all integers are distinct)\\
    \textbf{Step 1:} Let A be the array storing S. Create array B of length U. Set B to zero.\\
    \textbf{Step 2:} For $i\in [1,n]$; Set x to A[i], Set B[x] = 1\\
    \textbf{Step 3:} Clear A, For $x\in[1,U]$; If B[x] = 0 continue, otherwise append x to A
    \subsubsection{Analysis}
    Step 1 and 3 take O(U) time, while Step 2 O(n) time. Therefore running time is O(n + U) = O(U).

    \section{Random}
    RANDOM(x, y) returns an integer between x and y chosen uniformly at random

    \section{Data}
    \subsection{Data Structure}
    Data Structure describes how data is stored in memory.
    \subsection{LinkedList}
    Every node stores pointers to its succeeding and preceding nodes (if they exist). The first node is called the head and last called the tail. The space required for a linkedlist is $O(n)$ memory cells. Starting at the head node, the time to enumerate over all the integers is $O(n)$. Time for assertion and deletion is equal to $O(1)$
    \subsection{Stack}
    The stack has two operations; Push (Inserts a new element into the stack), Pop (Removes the most recently inserted element from the stack and returns it. Since a stack is just a linkedlist, push and pop use $O(1)$ time.
    \subsection{Queue}
    The queue has two operations; En-queue (Inserts a new element into the queue), De-queue (Removes the least recently used element from the queue and returns it). Since a queue is just a linkedlist, push and pop use $O(1)$ time.

    \section{Dynamic Arrays}
    \subsection{Naive Algorithm}
    \textbf{insert(e):} Increase n by 1, initial an array A' of length n, copy all n-1 of A to A', Set A'[n]=e, Destroy A.\\
    This takes $O(n^2)$ time to do $n$ insertions.
    \subsection{A Better Algorithm}
    \textbf{insert(e):} Append e to A and increase n by 1. If A is full; Create A' of length 2n, Copy A to A', Destroy A and replace with A'\\
    This takes $O(n)$ time to do $n$ insertions.

    \section{Hashing}
    The main idea of hashing is to divide the dataset S into a number m of disjoint subsets such that only one subset needs to be searched to answer any query.
    \subsection{Pre-processing}
    Create an array of linkedlist($L$) from 1 to $m$ and an array $H$ of length $m$. Store the heads of $L$ in $H$, for all $x\in S$; calculate hash value ($h(x)$), insert $x$ into $L_{h(x)}$. We will always choose $m=O(n)$, so $O(n+m)=O(n)$
    \subsection{Querying}
    Query with value $v$, calculate the hash value $h(v)$, Look for $v$ in $L_h(v)$. Query time: $O(\mid L_{h(v)}\mid)$
    \subsection{Hash Function}
    Pick a prime $p$; $p\geq m$, $p\geq$ any integer $k$. Choose $\alpha$ and $\beta$ uniformly random from $1,\ldots,p-1$. Therefore: $h(k)=1+(((\alpha k+\beta)\mod p)\mod m)$
    \subsubsection{Any Possible Integer}
    The possible integers is finite under the RAM Model. Max: $2^w-1$. Therefore $p$ exists between $[2^w,w^{w+1}]$.
    \subsection{Timing}
    Space: $O(n)$, Preprocessing time: $O(n)$, Query time: $O(1)$ in expectation
    
    \section{Week 3 - Extra}
    When using `Direction 1: Constant Finding' setting $c_1$, always set it to match the coefficent on the LHS so that you can cancel.\\
    When trying to get a contradiction, try and isolate an $x \cdot c_1$ on the RHS, where $x \in \mathbb{Z}$, such that an expression that contains $n$ is $\leqslant x \cdot c_1$\\
    Make judicious use of the $max$ function when adding functions together
    If $f_1(n) + f_2(n) \leqslant c_1 \cdot g_1(n) +c'_1 \cdot g_2(n) \leqslant max\{c_1 , c'_1 \} \cdot (g_1(n) + g_2(n))$, for all $n \geqslant max\{c_2, c'_2\}$.\\
    
    \section{Week 4}
    \subsection{The Master Theorem}
    
    Let $f(n)$ be a function that returns a positive value for every integer $n>0$. We know:
    
    \begin{align*}
        f(1) & \leqslant c_1\\
        f(n) & \leqslant \alpha \cdot f(\lceil n / \beta \rceil) + c_2 \cdot n^{\gamma} \text{ for } n \geqslant 2
    \end{align*}
    
    where $\alpha, \beta, \gamma, c_1$ and $c_2$ are positive constants. Then:
    
    \begin{itemize}
        \item If $log_{b} \alpha < \gamma$ then $f(n) = O(n^\gamma)$
        \item If $log_{b} \alpha = \gamma$ then $f(n) = O(n^\gamma \cdot log(n))$
        \item If $log_{b} \alpha > \gamma$ then $f(n) = O(n^{log_\beta(a)})$
    
    \end{itemize}

    %\section{Week 5}
    %TODO

    \section{SE Set 3}
    Find out how many times a recurrence takes to terminate, and then proceed to eyeball the time complexity
    
\end{multicols}

\end{document}

